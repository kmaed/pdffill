%#! lualatex
\documentclass[11pt]{bxjsarticle}

\usepackage{tikz}
\usetikzlibrary{positioning}
\usepackage{luajalayout}

\defaultfontfeatures{Ligatures=TeX}

\setrftovf[scale=.95]{IPAexMincho:+jp90}{rmfamily}
\setrftovf[overwrite,exclude=\jarange]{TeX Gyre Termes}{rmfamily}
\setrftovf[scale=.95]{M+ 1c-medium}{rmfamily-B}
\setrftovf[overwrite,exclude=\jarange]{TeX Gyre Termes/B}{rmfamily-B}
\setmainfont[BoldFont=vf:rmfamily-B]{vf:rmfamily}

\pagestyle{empty}

\begin{document}

\begin{center}
  \Large ほげほげ申請書
\end{center}

\begin{flushright}
  平成\hspace{3em}年\hspace{3em}月\hspace{3em}日
\end{flushright}

\noindent ほげほげ協会 会長 殿

\vspace{1em}

下記の通りほげほげすることを申請します。

\vspace{1em}

\begin{center}
  \begin{tikzpicture}[node distance=0pt]
    \draw (0, 1.2em) -- ++ (20em, 0);
    \node at (3em, .6em) {\footnotesize フリガナ};

    \draw (0, 0) -- (20em, 0);
    \node at (3em, -.8em) {申請者氏名};

    \draw (0, -1.6em) -- ++ (20em, 0);
    \node at (3em, -2.4em) {会員番号};
    \foreach \x in {8,10,...,18} {
      \draw (\x em, -1.6em) -- ++ (0em, -1.6em);
    }

    \draw (0, -3.2em) -- ++ (40em, 0);
    \node[right] at (0, -4em) {ほげほげするに至った理由};

    \draw (0em, 1.2em) -- (0em, -30em);
    \draw (6em, 1.2em) -- (6em, -3.2em);
    \draw (20em, 1.2em) -- (20em, -3.2em);
    \draw (40em, -3.2em) -- (40em, -30em);
    \draw (0em, -30em) -- (40em, -30em);
  \end{tikzpicture}
\end{center}

\end{document}
